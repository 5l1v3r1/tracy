\documentclass[a4paper]{article}

\author{Jurriaan Bremer \and Merlijn Wajer \and Bas Weelinck}
\title{Soxy: Transparent SOCKS proxy using \textit{ptrace(2)}}

\usepackage{hyperref}
\usepackage[utf8]{inputenc}

\begin{document}
\maketitle

\section{Soxy: Transparent SOCKS proxy using \textit{ptrace(2)}}

\subsection{Proxies}
In computer science, a \textbf{proxy} is an intermediate server that
accepts packets from a host and forwards them to another host,
therefore giving the sending host more privacy (that is, the
receiving host gets the IP address from the intermediate server).
Software can use a proxy server knowingly or unknowingly.

An application can be configured to connect through a proxy. For example, in
the popular web browser Firefox it is possible to specify how Firefox connects
to the internet, and one can directly tell Firefox how to do so.
It is possible to tell Firefox which credentials to use for which type of proxy,
if any at all.

In this case, the software explicitly has to implement functionality to use a
proxy. It is also possible to have software use a proxy even if it hasn't
implemented specific proxy functionality: by making the usage of a proxy
completely \textit{transparent} to the software.

By transparent we mean that the software unknowingly
\footnote{And perhaps even unwillingly} uses a proxy.

\subsubsection{Intercepting calls}

To transparently have a program use a proxy, a proxy program like \textbf{soxy}
has to intercept and modify the \textit{socket} function calls.

There are again several ways of intercepting and/or changing the socket function
calls.

A popular implementation of a transparent SOCKS 5 proxy \textbf{tsocks}
\footnote{We believe \textbf{torsocks} does something similar}
utilises the \textit{LD\_PRELOAD} feature on POSIX-compatible systems to
intercept the socket function calls and replace them with its own.

We propose using the POSIX \textbf{ptrace} feature to transparently have
software use a proxy.

\subsection{SOCKS 5}

SOCKet Secure (SOCKS) is an Internet protocol that routes network packets
between a client and server through a proxy server. SOCKS5 additionally provides
authentication so only authorized users may access a server.
\footnote{https://en.wikipedia.org/wiki/SOCKS}
\footnote{https://tools.ietf.org/html/rfc1928}

\subsection{ptrace}

\textbf{ptrace(2)} is a system call which provides a means to observe and
control the execution of another process by intercepting (all) the system calls
of said process.

This way it is possible to intercept \textbf{specific} system calls related to
networking. In theory, one could turn the data being send by the software into
proper SOCKS 5 packets and relay those packets to the proxy server. The same
could be done for incoming data (namely: extract the data from the SOCKS 5
packets and pass it to the software).

An additional advantage of using \textbf{ptrace} is that ptrace also allows
attaching to a running process in which case we could have a program
transparently start using our proxy without having to restart said program. This
does raise some serious complications, but theoretically it is possible.

\subsection{Features, limitations and goals}

We will strive to implement the following features:

\begin{itemize}
\item Support for TCP. (blocking and non-blocking)
\item Support for IPv4.
\item Support for rule-based proxies. For example, do not use a proxy for
host \textbf{foo} with port \textbf{bar}.
\end{itemize}

If time permits us, we will also implement:

\begin{itemize}
\item Support for IPv6.
\item Support for UDP.
\end{itemize}

In addition to these features, we will focus on maintainability and readability
of the implementation, rather than efficiency and speed. Typically proxies are
not doing much more than wrapping data in packets and relaying them to the proxy
server, and the speed of a proxy depends mostly on the speed on the network the
machine is using and not on the speed of the CPU.

We will also keep the code well documented and create a \textbf{man} page. \\

We will not focus on attaching to a running program and then have it use our
proxy functions. % kut zin
Instead, a program that will use said transparent proxy has to be started by our
proxy implementation by passing it as an argument to said proxy. For example:

\begin{verbatim}
soxy -v -- firefox -private
\end{verbatim}

Will run the program (Firefox) with the additional argument ''-private''.

\end{document}
