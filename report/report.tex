\documentclass[a4paper, 10pt]{report}
%\documentclass[a4paper, twoside, 10pt]{report}
%\documentclass[a4paper, twoside, 10pt, twocolumn]{report}

\author{Jurriaan Bremer \and Merlijn Wajer \and Bas Weelinck}
\title{Tracy: UNIX system call tracing}

\usepackage{graphicx}
\usepackage{listings}
\usepackage{hyperref}
\usepackage{appendix}
\usepackage{subfig}
\usepackage{wrapfig}
\usepackage{color}
\usepackage[utf8]{inputenc}

\definecolor{MidnightBlue}{rgb}{0, 0, 0.95}
\lstset{language=C,
    showspaces=false,
    showstringspaces=false
    basicstyle=\ttfamily,
    stringstyle=\color{MidnightBlue}\ttfamily,
    emphstyle=\color{red}\bf
}

\begin{document}
\maketitle

\nocite{*}
\bibliographystyle{plain}

\begin{abstract}
Presented is a uniform interface to trace the behaviour of programs
by means of the system calls they perform. Tracing by the user is done without
regard to kernel version, operating platform or processor architecture.
The interface, called Tracy, provides a means to watch, modify, augment
and restrict program execution in a controlled environment.
\end{abstract}

\tableofcontents

% TODO:
%
% - PTRACE_O_TRACE{EXEC,VFORKDONE,EXIT}
% - Noemen dat ptrace op een specifiek pid werkt altijd (is opzich logisch,
%       staat nergens)


% PTRACE_TRACE_{FORK,VFORK,CLONE} is a mess. If we simply want to be called on
% clone() in case of a clone event, we must disable fork and vfork; otherwise it
% may also instead return fork or vfork events. From the man page:
%
% This  option  may not catch clone calls in all cases.  If the tracee calls clone
% with  the  CLONE_VFORK  flag,  PTRACE_EVENT_VFORK  will  be   delivered   instead   if
% PTRACE_O_TRACEVFORK  is set; otherwise if the tracee calls clone with the exit sig‐
% nal set to SIGCHLD, PTRACE_EVENT_FORK will be delivered if PTRACE_O_TRACEFORK is set.

% Gebruiken PTRACE_O_TRACESYSGOOD voor performance.

% execve under ptrace:
% When one thread in a multithreaded process calls execve, the kernel destroys all other threads in
% the  process,  and  resets  the thread ID of the execing thread to the thread group ID (process ID).
% (Or, to put things another way, when a multithreaded process does an execve, at completion of the
% call,  it  appears  as though the execve occurred in the thread group leader, regardless of which
% thread did the execve.)  This resetting of the thread ID looks very confusing to tracers:

% Misschien pre/post renamen naar syscall entry en exit?

% XXX END NOTES XXX

%
% ---- STUFF -----
% socketcall
% KISS
% Add SOCKS5 somewhere. (Probably related work)
% UDP in the current SOCKS5 protocol
% UDP over ssh ( :-( )
% sshfs ergens noemen?
% ---- STUFF -----

\chapter{Introduction}

% AANTREKKELIJKE OPENING
Programmers spend a rather large part of their time finding mistakes in their
computer programs. Programmers usually refer to this routine as debugging.
Programmers employ various tools to aid them in debugging programs.

Several kinds of debugging exist. The most common debugging method is
focussed around crashes in a program. Other types of debugging include
performance analysis and analysis of system calls performed by the program.
A system call is a method for a program to request a service from an operating
system's kernel. The kernel of an operating system is the main component
of most operating systems; it is a bridge between the hardware and programs.

% XXX: Nog niet helemaal klaar, hier weer iets meer over system calls en wat
% er allemaal mee gedaan wordt? Het nut van het tracen van system calls in
% debuggers, e.d.

Inspecting (and modifying) the system calls performed by a program
is done programmatically using the \textbf{ptrace} system call.
A program inspecting the
% TODO: ptrace, ptrace, textbf{ptrace}, etc? Nu clarifyen?
system calls of another program is said be to be ``tracing'' the other
program. The program that is being traced is called the ``tracee'' and the
program tracing the ``tracee'' is called the ``tracer''.
Tracing is a limited definition in the sense that the term does not
mention the possibility to modify the program that is being traced.
Using \textbf{ptrace} the tracer can also modify the tracee by
reading and writing to the memory of the tracee or even changing the
system calls made by the tracee.

% AANLEIDING / MOTIVATIE

\section{Motivation}
The ability to modify the program memory and registers
\footnote{Changing cpu registers allows for system call modification}
allows for a great variety of applications.
A possible application is the so called ``jailing''
of programs - only allow system calls that satisfy certain requirements
and modify the arguments to some system calls to effectively
restrict the access of a program to the system.

Other applications are transparently routing or inspecting all the
network traffic of a program.
Being able to modify the system call arguments and return value also allows
for extensive stress testing of programs, one could fake being out of available
memory for example; or even randomly reject or fail system calls.

Tracing programs is done using the \textbf{ptrace} system call on
most\footnote{If the UNIX system supports ptrace.} UNIX systems. % Het is namelijk niet POSIX
Using ptrace directly to trace a program has several downsides: the ptrace
interface is not very programmer friendly: the interface is not standardised
(ptrace is not part of the POSIX\cite{posix} standard) and ptrace is not
architecture agnostic, meaning a tracer requires architecture specific code.

% Centrale vraag
% - Hoe kan system call tracing worden verbeterd met betrekking tot API
%   transparantie voor kernel versie en architectuur op Linux?
%
% Deel vragen:
%   - Welk effect heeft de architectuur op het interpreteren van system calls?
%   - Welk effect heeft de architectuur op memory access?
%   - Welke problemen zijn er met de ptrace API.
%   - Wat voor effect heeft ptrace op de performance van programma's?
%   - 

% TODO: Programmer friendly? Perhaps emphasis?
We researched the viability of a cross architecture, programmer friendly
system call tracing and modification framework for the Linux kernel.
During this research, we put emphasis on a few key areas.
% Opmerken dat het niet alleen voor Linux hoeft te zijn?

\subsection{Problems with ptrace} % TODO
First of all, we wanted to solve problems introduced by utilising ptrace
directly. As we have previously noted, a particular problem with ptrace is that
the ptrace interface to the programmer differs per platform. This means that
ptrace code written for Linux will most likely not work on another UNIX operating
system such as FreeBSD.

Different UNIX platforms also support different ptrace features. FreeBSD offers
a ptrace option to quickly read or write a large amount of memory of the
tracee, whereas Linux does not support reading more than a processor word at a
time. On the other hand, Linux has an ptrace option to automatically trace
any child processes created by a tracee; FreeBSD currently does not offer such
an option (Section \ref{safe-fork}). % strace manpage ref?

% Dit is meteen wat diep duiken in de stof.
The system call invocation differs per computer architecture.
The assembly instruction to invoke a system call usually differs.
Kernels like Linux and FreeBSD sometimes even support several ways to invoke
system calls. ptrace does not provide a way for the tracer to detect how
the tracee invoked the system call. This is problematic because the meaning
of system call numbers differs per system call invocation thus leading to
inevitable amiguity about the current system call being performed (Section
\ref{secure-abi}).

% TODO: Dit uitbreiden
Due to ptrace being a very low-level system call, which operates on assembly
level, it is not possible to simply ``change the return value'' of a system
call. Each instruction set has its own registers and on top of that, operating
systems often have different uses for each register.

One would have to find out what register this is on each platform,
keep track if the current ptrace event describes the start or end of a system
call\footnote{If the process is about to execute a system call or has just
executed a system call}, let alone perform some extra architecture-specific
calls to ensure that the result is properly set.\footnote{ARM requires
the programmer to call ptrace with the \textit{PTRACE\_SET\_SYSCALL} command
to make the changes to the system call register permanent.}

Finally, ptrace offers no mechanism to recieve events of specific system calls,
rather than on every system call. This leads to performance issues (Section
\ref{ptrace-perf}).

\subsection{LD\_PRELOAD}
% 
% % There are several ways to transparently intercept calls to a library or
% % system calls. LD\_PRELOAD is one such method to intercept calls to a library.
% % This should not be confused with intercepting system calls; LD\_PRELOAD
% % cannot intercept system calls; this is a reason not to use LD\_PRELOAD.

\textit{LD\_PRELOAD} is a way to prioritize loading of certain libraries
while creating and loading a process. The \textit{LD\_PRELOAD} environment
variable controls which libraries are loaded into a process before all
other libraries are loaded.

One can use \textit{LD\_PRELOAD} to load a shared object (a library)
into a process, before any other libraries (such as glibc) are loaded;
this makes it possible to (transparently) override functions typically
provided by other libraries.

This mechanism can be used (amongst other) to create a proxyfier.
Such an ``injected'' library could provide its own ``read'' and
``write'' functions which will be called by the program in favour of
the default read and write functions.

One common proxyfier, \textit{tsocks}, is an application that loads
other applications with their own library that overwrites specific
network functionality, thus allowing transparent network routing.

LD\_PRELOAD relies on the dynamic loader and will only have any effect
if the programs have to make use of the libraries being loaded.

A downside to this approach is that \textit{LD\_PRELOAD} simply does not work
on all programs; some programs do not use glibc methods and instead perform
system calls directly, for example by directly using assembly in their program.
Other programs are simply statically linked. %TODO Uitleggen waarom statically
% linked uitmaakt, en wat het is?

There are some upcoming languages that do not even use glibc at all - languages
like Go \cite{golang} talk to the kernel directly and thus perform their
system calls by calling the kernel directly instead of
relying on (g)libc functions which would in turn call the kernel.

\textit{LD\_PRELOAD} is not a viable solution when one wants to transparently capture
all (specific) system calls of a process, independent on what kind of
userspace libraries the process uses.

\section{Structure of This Document}


\section{Terminology}

% System calls zijn bold

\chapter{Tracy and ptrace}

\section{Introduction to Tracy}
Several UNIX and UNIX-like systems support the \textbf{ptrace} system call.
This system call allows one process to ``control'' another process - from now on
called \textit{tracee}, enabling the controlling process to stop and inspect
the \textit{tracee}, as well as writing to its memory. \textbf{ptrace} is
mostly used by tools that aid debugging of software such as GDB\cite{gdb}.
\textbf{ptrace} provides two ways of controlling a process
- either by controlling all the system calls made by the process or by stopping
the tracee after each instruction. The latter option has a serious effect on the
performance of the tracee and is of no use for system call tracing.
% XXX: Why impact on performance?

Tracy is a library that uses \textbf{ptrace} to trace the system calls of
a process. Tracy can inspect, modify and even inject system calls. Being able to
modify system calls gives the controlling process the ability to change
arguments of system calls before system calls are executed - and even the
ability to ``deny'' system calls by changing the system call to a harmless
one such as \textbf{getpid}.

Obvious use cases for modifying system calls is a process
called ``sandboxing'' - intercepting and changing special system calls like
\textbf{open} and prepend (if not already in place) a specific path to the
arguments such that the tracee cannot open files outside a specified path.

\textbf{ptrace} is very platform and operating system dependent.
The Tracy API strives to be platform and operating system independent by
applying architecture and operating system dependent hacks and fixes
``behind the scenes''.

We have tried to design Tracy to be as portable as possible - meaning that code
for computer architectures such as ARM and x86 should be nearly
(if not completely) the same.

% TODO: Is dit nodig? Moet dit niet ergens anders?
% There are some caveats (Section \ref{});
% but some of these are simply not fixable without extending the focus
% of Tracy to such an extent that Tracy would no longer be simply
% a library to trace and inject system calls.

% Is dit nodig? Kan dit niet ergens anders?
Tracy elegantly works around most architecture specific issues and implements
functionality required to trace programs on several operating systems.
% XXX: weten mensen al dat dit linux en bsd limitations zijn?
(This includes fast memory read/write on Linux and in the future safely tracing
children on BSD)

Tracy introduces an API that allows a programmers to hook into specific system
calls, rather than every system call as is the case with ptrace.\footnote{Of
course, under the hood Tracy will have to handle every system call, whether
the system call is hooked or not. We propose a fix to this problem in
Section \ref{ptrace-perf}} Tracy also has both a synchronous and asynchronous
injection API (Section \ref{syscall-inject}) and Tracy keeps
track of the state of a system call - if the event is an event generated before
a system call is executed; or after a system call is executed.

% Arguments aanpassen. Dit gaat alleen om de buffer. Anders kan je gewoon de
% waarde overschrijven.
It can be useful to data pointed to by a system call argument of tracee.
To transparently and safely change these additional hacks are required:
both changing the contents directly as well as validating the value when the child
can still change the value are bad ideas.
% Bad ideas is een overdrijving.

The first interferes with the normal execution of the process
(although this may be exactly what the programmer wants) and the second is
sensitive to race conditions. A solution is presented in
\cite{Noordende_asecure} and the feature in Tracy is planned (Section
\ref{memory-share}).

Tracy has functionality to allocate memory in the tracee which
the tracee cannot write but only read; thus allowing the tracer
to copy arguments to that memory space, validate and change them as
necessary; finally change the pointer in the arguments to point to the memory
space and continue the system call as normal.
% XXX: Moet je nog wel mprotect enzo gaan doen. SYSV Shmem is een
% andere oplossing.

\section{System call tracing, modification and injection}

\textbf{ptrace} suspends the tracee and signals the tracer right before
the tracee executes a system call as well as just after the tracee has
executed a system call. When the tracee is stopped before a system call,
we say that the tracee is in \textbf{pre} system call state; if the system
call has executed and the tracee has once again
been suspended we say that the tracee is in \textbf{post} system call state.
While the tracer is inspecting the tracee, the execution of the
tracee is paused. (Figure \ref{fig1})


\begin{figure}
\includegraphics[scale=0.6]{ptrace.png}
\caption{Ptrace control flow}
\label{fig1}
\end{figure}

When the tracee is suspended, the tracer can read and modify registers as well
as read and write to the memory of the tracee. Combining these two features,
the tracer can:

\begin{itemize}
\item Modify the system call (number) that is executed.
\item Modify the arguments to system calls of the tracee.
\item Modify the instruction pointer (commonly called program counter), allowing
    the tracer to resume the executing of the tracee at an entirely
    different set of instructions.
\item Inject system calls, by modifying the instruction pointer and changing the
    system call number.
\item Modify the return value of the system call.
\end{itemize}

\section{Memory access}

To be able to fully modify a tracee, the tracer needs to be able to
read and write the tracee's memory. If the tracer is to change
arguments that point to a specific page in the tracee's memory - like a string,
the tracer needs to be able to write to the tracee's memory - or
share a small part of the tracers own memory with the tracee.

Changing the memory of the tracee is usually not a good idea, as that could
interfere with the tracee's execution.
Instead, it would be a better idea to allocate a few pages in the
child and copy the data to those pages, change the argument to point to the new
pages and continue the system call. After the system call has completed, the
pages can be freed again. This way the child's original arguments are left
untouched, and for a reason: the child may want to re-use the memory later on.
% XXX: Ook hier is ``arguments'' de data waar de pointer argumenten naar
% wijzen, moet een duidelijke distinctie krijgen.

Allocating pages in the tracee as described above is problematic if one wants
to make sure the tracee does not change the memory contents. ptrace suspends
only one process and each thread on Linux is a seperate process. Threads share
memory with other threads in the thread group and can thus write to the same
memory. If one thread is suspended, another thread can still write to the newly
allocated pages and change the data stored in the pages just before the system
call is executed and after the tracer has written data to the page. To safely
share memory, see Section \ref{memory-share}.


%TODO: Pages allocaten in child (wat voor beide RW is) is racy;
%zie future work memory sharing oplossing.

\section{Tracing children}

% XXX: Hier staat nog geen rfork() bij.
A tracee can spawn children by calling the \textbf{fork}\footnote{among
    other system calls: vfork, rfork (BSD), clone (Linux)} system call. \textbf{ptrace}
does not automatically trace children of a tracee.

In certain occasions, one may want to trace all the children of a tracee.
To do this, Linux supports specific \textbf{ptrace} options to trace
the tree of children spawned by the tracee as well (Section
\ref{syscall-trace}).

However, other platforms do not support these options and tracing
children of the tracee can be somewhat problematic.
% XXX: Wat is het probleem dan? Zie strace quote hieronder.
We provide a solution
to this in Tracy such that any children spawned by the tracee will not
run uncontrolled on any platform in Section \ref{safe-fork}.
%\footnote{strace does not
%immediately trace children on non-Linux platforms:
%% Cite strace hier weer, man page citen? Waar komt de quote vandaan (man page)
%\begin{quote}
%    On non-Linux platforms the new process is attached to as soon as its pid is
%    known (through the return value of fork in the parent process).
%    This means that such children may run uncontrolled for a while
%    (especially in the case of a vfork), until the parent is scheduled
%    again to complete its (v)fork  call.
%\end{quote}
%}

\begin{figure}
\includegraphics[scale=0.4]{ptrace2.png}
\caption{One system call with ptrace}
\label{fig2}
\end{figure}


\chapter{Implementation}

\section{Tracing a process}

Tracing a process can be done in two different ways. A process can either attach
to another running process, or create a child process and consequently use
% consequently? Weird.
\textbf{ptrace} in the child to make the parent (the original process, soon to
be the \textit{tracer}) trace the child after which the child will
call \textbf{execve} to run another program.

\subsection{Fork and trace}

As explained in the previous section, the parent issues a \textbf{fork} call.
The child then performs a \textbf{ptrace} call, with the \textit{PTRACE\_TRACEME} argument.

% XXX: Het is hier niet duidelijk dat dit SIGTRAP alleen geraised wordt door
% het tracy child en niet automagisch door ptrace oid.
The child then sends itself a \textit{SIGTRAP} signal; because the child is being
% XXX: de because hier is vreemd.
traced by its parent, the child is then suspended until the parent performs
a \textbf{wait} followed by a call to ptrace to continue the process with the
\textit{PTRACE\_SYSCALL} argument to \textbf{ptrace}.
The parent can inspect and change the child before it issues
the continue call. The code demonstrating this can be found in
Appendix \ref{appendix:createtrace}

\subsection{Attaching}

% TODO: How does one attach
%
% XXX: Noemen dat we attach ook gebruiken in safe-fork

\section{System call tracing}
\label{syscall-trace}

% TODO: Mist nog wat diepte. Hoe krijgen we precies notifications? (waitpid)
% XXX: Add (PTRACE_O_TRACESYSGOOD).
Tracing system calls with \textbf{ptrace} is done by issuing ptrace on the
tracee with the \textit{PTRACE\_SYSCALL} option, this will make the tracee
suspend on entry (pre) and exit (post) of the next system call. Resuming a
suspended tracee is done with the same option. Tracy uses this option on
% XXX: Resume ... done with the same option? VAAG
all the tracees.

Tracy keeps track of each child; it has to store information about the state
of the child: will the next stop be a system call entry or exit, or are we
currently injecting a system call? Aside from those two states, Tracy also
has to do some other internal bookkeeping.
% XXX: Zonder voorbeeld/ref dit ^^ gewoon niet noemen.
% vb: injected nr, etc

% Meer is niet nodig denk ik, er staat ook een hoop in het event system.

\section{System call modification}

Modifying system calls can be done at two points: before and after the
system call is executed. Changing the system call ``after'' it has been executed
does not undo the effect of said system call, but it allows the tracer to change
result value of the system call.
\footnote{The tracer can of course change every register and the memory of the
tracee at any point when the tracee is suspended.}
Changing values before the system call is executed is more exciting, as it
allows the tracer to actually \textit{change} the system call. For example, by
modifying the register that stores the system call number, it is possible to
execute a completely different system call. This serves two purposes:
``denying'' system calls by changing the system call register to a harmless
system call such as getpid and ``injecting'' system calls by changing the
system call number and later on the instruction pointer (See Section
ref{syscal-inject}).

% TODO: Tracy specifiek stuk hier? Methode namen hoeft niet, maar wellicht wel
% hoe wij het doen?

\section{Event system}
\label{event-system}

Tracy uses event-based system to report on activities on a tracee. This is
particularly fitting as the \textbf{waitpid} call blocks until an event
occurs. As a result Tracy blocks until a new event occurs.

Each signal and system call (pre and post) is an event.
The Tracy function \textit{tracy\_wait\_event} waits for a action on
(a specific or) any child and returns a new Tracy event.

% XXX: Beetje diep ineens... Verhaal mist nog wat voordat het hiernaartoe
% springt.
Aside from events for signals and system calls, Tracy also exposes another
event; an internal event which describes an internal Tracy event; this event
is required for asynchronous injection and may be used for other features in
the future.

\subsection{System call hooks}

Users of the Tracy library can hook specific and even all system calls. For
each system call, the user provides a callback which will be called when
the related system call occurs. The callback is provided with full access to
event. The return value of the callback determines the action that Tracy
will take. A callback for signals also exists as well a callback default
callback that is called for all the system calls that are not hooked.

\section{System call injection}
\label{syscall-inject}

% XXX: Vreemde plek.
Tracy supports injecting system calls into any process that is being traced.
The injection of a system call is the process of executing a system call in
a specified tracee, without requiring said tracee to explicitly call the
system call.

Theoretically a system call can be injected at any point during
the execution (as discussed in Section \ref{instant-inject}),
Tracy however can currently inject system calls in a pre and post
system call state. In other words, one can currently only inject system calls
when a tracee is stopped due to the tracee executing a system call; not if
the tracee is recieving a signal.

Transparent injection of system calls is not possible, in the most strict sense.
As most system calls indirectly affect the tracee (with the exception of system
calls like \textbf{getpid} and \textbf{fstat}), the tracee will usually
be left in a modified state. However, to perform system call injection as
transparently as possible, we ensured that the registers before the injection
are identical to the registers after the injection.

Injecting system calls relies on that fact that the program counter (or
instruction pointer) of the tracee can be modified. If the length of the system
call instruction is known, it is possible to ``jump back'' to the system call
instruction, see Figure \ref{fig:asm-jump}.
% TODO: Add figure with asm + int 0x80 and arrows.


The process of injecting a system call differs per system call state (pre or
post).

\begin{figure}
    \centering
    \subfloat[Injection from a pre state]{\label{fig:pre-inj}\includegraphics[width=0.4\textwidth]{pre.png}}
    \hspace{1em}
    \subfloat[Injection from a post state]{\label{fig:post-inj}\includegraphics[width=0.4\textwidth]{post.png}}
    \caption{Injection for pre and post system call states}
    \label{fig:injection}
\end{figure}

\begin{figure}
    %\includegraphics[scale=1.0]{asm-jump}
    \caption{} % TODO
    \label{fig:asm-jump}
\end{figure}

% TODO: Jump naar pre en post uitleg.

\subsection{Pre-syscall injection}

As seen in figure \ref{fig:pre-inj}

\subsection{Post-syscall injection}

As seen in figure \ref{fig:pre-inj}


\subsection{Asynchronous injection in Tracy}

Tracy supports asynchronous injection. Typically, when injecting a system call
in a tracee, Tracy waits for the system call that is being injected to complete
and then restores the tracee to its previous state. This process can be split
into two stages, readying the tracee to perform a system call and restoring
the tracee to its original state. No system call returns instantly and it
may be favourable to perform other tasks rather than waiting for the system
call to complete.

Asynchronous injection is a way to inject system calls that allows Tracy to
handle other system calls while Tracy is waiting for the injected system call
to finish, this can be useful as some system calls can take quite some time.
Once process $A$ has finished its (injected) system call, Tracy will store
the return value and restore the process to its original state.

Functionally this makes no difference to the injection of a system call, but
it does allow Tracy to handle the requests of other (suspended) processes
instead of simply waiting for the system call of the tracee to finish.

While asynchronous injection is usually preferred, synchronous injection is
supported by Tracy as well and even used in Tracy internally where
asynchronous injection simply isn't worth the extra work and effort.

\section{Memory access}

% TODO: process_vm_{read,write}v noemen.

Any program that uses the \textbf{ptrace} API will probably want to access the
traced process' memory as well. Reasons can range from simply dumping pointer
contents to monitoring, changing or even injecting new system calls with
completely stand-alone data.\

% Furthermore, in case of IO intensive operation this access should be fairly
% fast as well.
% XXX: Dit is niet echt goed geplaatst.

Unfortunately as with most ptrace functionality, memory access standards
differ from operating system to operating system and are also
affected by architectural quirks. Tracy hides these differences and
provides a uniform and fast way to access memory.

\subsection{On Linux}
On Linux Tracy achieves fast memory access by employing a feature of the
``/proc'' filesystem that was designed to work in unison with
\textbf{ptrace}.

The classical way of accessing child memory on Linux was through the
\textit{PTRACE\_PEEKDATA} and \textit{PTRACE\_POKEDATA} operations.
The downside of this method is its dependence on C's long type and the
amount of data that can be transferred.
These ptrace operations are used to respectively read and write individual
processor words.

% XXX: Hoezo weten we dat niet? Gewoon claimen of niet zeggen IMHO
On Linux the "long" type has, as far as we know, the same size as the
processor word size, so 32 bits on a 32-bit architecture and
64 bits on a 64-bit architecture.

This is where the trouble starts, not only is it not possible to simply read
or write single bytes of memory, whenever one wants to access say 4kb of
memory, depending on the architecture, this may require more than 1024
calls to ptrace. Needless to say the amount of time spent in context switches
and kernel code is enormous.
% XXX: Context switches, waar? Het child is suspended.

\subsubsection{Alternative: through /proc}
Linux provides another way to access memory of a process. The ``/proc''
filesystem, used by a lot of utilities to provide information about kernel
and process state can also be used to access arbitrary process memory,
given certain (security related) requirements are met.
Process information can usually be acquired by opening a folder
named ``/proc/$<pid>$'' where ``$<pid>$'' must be substituted with the process
identifier. In this folder several files, and folders, are contained which
provide information on things such as, threads, open files, memory maps and
memory contents.

The ``/proc/$<pid>$/mem'' file is the one used by Tracy to access child memory.
Opening this file will present child memory as one contiguous file which
can be modified using the usual IO operations.
% Contigous lare? What are the usual operations? Elaborate

This also means we can now write single bytes of memory without the need
of first reading a processor word. We can now read 4kB of memory with a
single system call or any arbitrary amount of memory provided we can allocate
a large enough buffer to contain the data.

One cannot simply open the ``/proc/$<pid>$/mem'' file of any
arbitrary process as this would cause a major security issue. The only
processes capable of successfully opening this file are either processes
tracing the target process, or super-user processes.

% XXX: Future work, memory mapped child access?
% /proc/<pid>/mem
% (PEEK|POKE)USER

\section{Signals}

Tracy obtains information about the signals directly from the
\textbf{waitpid} call. The waitpid call has an argument called
\textit{status} which contains the information about the signal that was
received. Retrieving the signal sent to the tracee is done using the
\textbf{WSTOPSIG} macro. \\

To pass a signal to a tracee, Tracy passes the signal number as data
argument to the ptrace \textit{PTRACE\_SYSCALL} api call. This will
resume the execution of the tracee with said signal. If the data is
left at zero, no signal is delivered. This makes it possible to
suppress certain signals by simply not passing them along to the tracee.


\section{Tracing children}

% TODO: This can't be empty

\subsection{Linux}

Linux 2.6 and onwards offer options to automatically trace all children
created by \textbf{fork}, \textbf{vfork} and \textbf{clone}.
Respectively: \textit{PTRACE\_O\_TRACEVFORK}, \textit{PTRACE\_O\_TRACEFORK},
\textit{PTRACE\_O\_TRACECLONE} flags are passed to \textbf{ptrace} to enable
tracing of children.
% TODO: Wat kort
% TODO: CLONE_UNTRACED vs PTRACE_O_TRACECLONE

\subsection{Safe execution of fork, vfork}
\label{safe-fork}

On operating systems that do not support the \textit{PTRACE\_O\_TRACE*} options,
another solution is required to ensure that all children are traced the moment
they are created. To achieve this, the \textbf{fork}
(as well as vfork and clone) system call must be executed in a
controlled manner.

To implement this feature, we make use of system call injection as well as the
ability to be able to allocate and write to pages in the tracee. We allocate a
page in the tracee and write a few lines of assembly that will safely execute
\textbf{fork} to the page. \ref{safefork-asm}
Once this the assembly has been written, we deny the initial
\textbf{fork} call by changing the system call to \textbf{getpid}.
In the POST-getpid callback, we change the Instruction Pointer
(or Program Counter) to the first instruction in the newly allocated page.


Once the tracee starts executing the instructions in this
page, the tracee will once again execute a system call -
\textbf{fork} in this case, and this time we allow the call to proceed.
However, after the \textbf{fork} has completed
and we have stored the result of the call (the process id), both the tracee
and the child execute the rest of the instructions in the page.

The only other instructions in the page are a busy while loop that
calls the \textbf{sched\_yield} system call.
Now that both the tracee and the child of the tracee are caught
in this while loop.

At this point we are still only tracing the tracee, but we can now use the
``attach'' mechanism of ptrace to attach to the child of our tracee. Once we are
tracing the child of the tracee as well, we can restore the Instruction Pointer
(or Program Counter) of each tracee to their original position
(that is, just after the original \textbf{fork} call) and allow both
processes continue their execution.

% picture of the process. Just use dot. (We also link to the code in the
% appendix)

% XXX: Signals e.d. hier noemen? Hoe we die gebruiken voor het verkrijgen
% van het pid bij vfork. (want de returned niet in het parent totdat het child
% dood is)

\subsection{Children with threads}

% Hoe werkt resource sharing? CLONE_VM, CLONE_FILES, unshare, etc.

% Introductie naar threads. Wat beschrijft een thread nu?


% Threads op Linux zijn processes die bepaalde resources sharen (CLONE_VM,
% CLONE_FILES).

% Waar wordt aangegeven wat ze delen? Hoe kan je dingen ``on delen''? (unshare)

% Hoe kan je in Tracy threads detecteren? (TID group, etc)

% We could also talk about PTRACE_O_TRACEEXEC and PTRACE_O_TRACEVFORKDONE
% and PTRACE_O_TRACEEXIT

\subsection{Caveats}

When the feature to automatically trace newly created children in Linux
is turned on, the tracer will recieve a SIGSTOP signal event for the tracee
that has just been created. This signal is not passed along to the tracee.
For this reason, Tracy suppresses the first SIGSTOP that is sent
to a newly create tracee if the automatically trace children option
is enabled.

\section{Performance in ptrace}
\label{ptrace-perf-problems}

When a tracee performs a lot of system calls, the overhead of \textbf{ptrace}
becomes very noticeable. The main issue is that for every system call the kernel
has to perform four context switches. The first switch is suspending the tracee
and replacing the tracee with the tracer (Tracy), then Tracy issues a few
(at least two) system calls (which require entering kernel mode),
the kernel then resumes the tracee (another context switch); once the system
call of the tracee is completed, the tracee is suspended again and the tracer
takes over (context switch number three), again performing at least two ptrace
system calls and finally the tracee is resumed again (context switch number four).

This entire process is repeated for every system call, be it the
inexpensive \textbf{getpid} system call or the fairly expensive
\textbf{clone} system call, they are all a lot slower for the tracee
due to this overhead of suspending and resuming the tracee.

The problem is that \textbf{ptrace} does not allow the reporting of specific
system calls only - the tracer has to handle every single system call event,
even if the tracer will take no other action than resume the tracee on most
of the system calls.

We present a (theoretical) solution to this problem in
Section \ref{ptrace-perf}.

% Misschien verkeerde titel; maar dit zijn dingen waar we tegen aan liepen.
% En dingen waar we op moeten letten.
% We hebben bij veel secties al een caveats subsectie...
\section{Caveats}

% TODO Dit mag niet leeg zijn.

\subsection{Clone}

The Linux specific \textbf{clone} system call is a bit problematic, for
several reasons. The clone system call interface differs per architecture;
on x86 clone has one argument extra compared to AMD64; and Linux on x86
also \textit{switches} two registers
\footnote{This can be observed in \textit{arch/x86/kernel/entry\_32.S}
and \textit{arch/x86/kernel/process.c}}.

 % XXX: Welke ook alweer

% x86: Args switched... Waarom? Misschien wel leuk om uit te zoeken.
% x64: Argument minder dan x86

\subsection{vfork}

% Problemen die we tegen kwamen met vfork? (Het returned niet een pid
% met safe-fork)

% FIXME title sucks
%\subsection{32 bit ABI on 64 bit}
%
%As of yet, Tracy does not support running 32 bit tracees with a 64 bit
%Tracy instance.


\chapter{Soxy}
\label{chapter:soxy}

In order to show the power and features of Tracy, as well as finding bugs and
improving the API, we have developed Soxy, a proxifier \footnote{An
application which tunnels internet traffic through another server} on top of
Tracy which works solely on existing binaries (that is, executables which do
not provide such functionality natively.)

Soxy has the unique ability to tunnel all (or a subset of) the network traffic
created by a process (either a child processes or a running processes) over a
proxy based on the SOCKS5 \footnote{\href{http://www.ietf.org/rfc/rfc1928.txt}
{RFC 1928 - SOCKS5 Protocol}} protocol.

The server which will be used as proxy server must be configured to accept and
tunnel SOCKS5 data. There are several proxy daemons available, including
\verb=ssh=.

\section{Python bindings}

% socketcall x86 vs native syscalls on x64 in caveats

Besides Soxy, which demonstrates an example usage of the Tracy library, we
also provide Python bindings, allowing rapid development of applications
which utilize Tracy. As a matter of facts, Soxy has been implemented using the
Python bindings, ensuring the correctness and efficiency of these bindings.

The Python bindings provide a similar, but even easier, API for using Tracy.

For example, a pre-syscall hook on \textit{write} which duplicates the write
system call (i.e. write the same string twice) looks like the following.

\begin{lstlisting}[language=Python]
def hook_write(child, event, args):
    if child.pre:
        child.inject(args.syscall, args)
\end{lstlisting}

\section{Soxy Internals}

\subsection{SOCKS5}

Complete documentation of the SOCKS5 algorithm can be found in
RFC 1928 \footnote{\href{http://www.ietf.org/rfc/rfc1928.txt}
{RFC 1928 - SOCKS5 Protocol}}. The global flow for TCP connections
is as follows.

\begin{itemize}
\item Application connects to the Proxy Server
\item Optionally the application performs some sort of authentication
\item A request containing the destination address is sent
\item Server replies with success or failure
\item Proxy connection has been established
\end{itemize}

After the proxy connection has been established, the protocol does not need
any further work, because all incoming and outgoing traffic goes through the
proxy server.

\subsection{Implementation}

Soxy implements the SOCKS5 protocol as follows, using hooks on the
\textit{socket} and \textit{connect} system calls. When a
socket is created, using the socket system call, Soxy determines whether the
socket is TCP or UDP based. Following, when the socket attempts to connect to
another machine, using the connect system call, Soxy injects a few system
calls in order to connect to the proxy server, authenticate, and establish the
proxy connection.

\subsection{Asynchronous Sockets}

\subsection{Proxifying UDP Traffic}

\chapter{Future work}

\section{Threaded tracer}

As the number of tracees for a single Tracy instance increases, the
likelyhood of Tracy becoming a bottleneck increases. Since Tracy is
single threaded, it is very possible a lot of tracees will have to wait
on Tracy to perform a system call. \\

We believe that it is possible to add multithreading support to Tracy where
each thread can perform its own \textbf{waitpid}. Access to certain
datastructures in Tracy will have to be guarded with semaphores, but
we think multithreading will be both viable to program and an effective
wait to increase the performance of Tracy with a lot of tracees.

\section{Secure ABI mixing}
\label{secure-abi}


% TODO: integrate https://en.wikipedia.org/wiki/W%5EX
% link to AMD ?
% link to arm? explain oabi and eabi?
% Explain ABI (before this chapter?)
On some platforms such as AMD64 and ARM Linux implements several ABIs.
These ABIs are sometimes invoked differently (as is the case with AMD64)
and it is not easy to tell what ABI is being used to issue the system call.
The problem here is that each ABI (sometimes) has different system call
numbers, which means that \textbf{read} on AMD64 ABI is not the same
as the 32 bit ABI on the AMD64 platform. Tracy currently always assumes
a x86\_64 ABI. This is not just a functional issue (not being able to
properly identify a 32 bit ABI system call) but a security problem because it
allows any tracee to ``fake'' Tracy into thinking it is doing system call $A$
where it is in fact doing system call $B$; Tracy will not call the proper hook
function and (if told to do so) may deny the wrong system call or even worse,
not deny the system call at all. \\

In a sense not being able to identify the system call ABI can be seen as a
limitation of the Linux \textbf{ptrace} API. It should be possible to
integrate support for identifying the system call ABI in the Linux kernel
by using a system that is already in place: the \textit{PTRACE\_O\_TRACESYSGOOD}
extension. If \textit{PTRACE\_O\_TRACESYSGOOD} is enabled, the kernal sets bit 7
of the signal being delivered to the tracer; this makes it easier to distinguish
between a normal \textit{SIGTRAP} signal and a trap caused by a system call of
a tracee. The kernel could expose (and set) another bit (or more) which would
indicate the ABI of the current system call.

However, a kernel change is not an immediate solution and would require all
users of Tracy to install a very recent Linux kernel or even apply a patch
to their current kernel. \\

Another solution to the problem of identifying the ABI of the current
system call would be to read the current assembly instruction. AMD64 uses a
different instruction to perform a system call than the 32 bit ABI; which
means we can use that instruction to differentiate the two ABIs. There are a
few caveats however: it is possible that the instruction in the program (memory)
and the instruction in the cpu pipeline are not the same; a program could use
another thread to intentionally wipe out or change the instruction that was
used to invoke the system call, which would result in Tracy reading a faulty
or fake instruction. \\

To prevent another thread in the process to change the instruction we can
use the \textbf{mprotect} system call to mark the memory as read and execute
only (not writable), which would make it impossible to change the contents of
the memory without calling \textbf{mprotect} to mark the memory as writable.
Calling \textbf{mprotect} from a tracee will however be noticed by Tracy,
upon which Tracy can take appropriate actions (such as allow the memory to be
written but mark it read and execute only after the write has occurred and make
sure the cpu pipeline has is flushed by calling jmp).

One could even go as far as cache the system call instructions of each tracee
so they do not have to be read again until a tracee changes its own
instructions.

Even though now yet supporting identifying the ABIs is a serious security issue,
it is also mainly important when someone wants to be completely sure that every
single system call is traced. Most programs also do not usually modify their
own instructions; nor do they create their own instructions (JITs do)
% XXX: JIT reference, Java is jit, ofc, firefox js etc
that perform 32 bit system calls in a 64 bit process.

\section{BSD Support}

Support for Tracy on the BSD variants is planned, but will require some
additional work. We have not extensively examined \textbf{ptrace} and
system calls in general on BSD variants, but a few things jumped out. \\

BSD uses \textbf{rfork} instead of \textbf{clone} and BSD also does
not support automatically tracing processes created by a tracee. While
we can emulate this effectively with Tracy's safe fork
Section(\ref{safe-fork}) it will still require some work.

Apart from this functional difference, there is also a difference in
how BSD variants treat the system call arguments. Arguments of a system
call are pushed onto the stack (the last parameter is pushed first).
\cite{int80h}

Alternatively, if Linux emulation is enabled, BSD variants (at least FreeBSD)
also support the Linux way of system calls, by passing the arguments in
registers in the same manner as Linux.

BSD also returns errors differently than Linux (although again, it supports
Linux emulation). BSD sets the ``carry flag'' upon failure, whereas Linux
returns a negative value upon error.

\section{Memory sharing between tracer and tracee}
\label{memory-share}

% Need to do some shm magic.

\section{Instant injection using signals}
\label{instant-inject}

An interesting addition to Tracy would be the ability to inject a system call
even when a tracee is not performing a system call: that is, when the tracee
is not stopped. We can simply stop the tracee by sending it a signal will
\textbf{kill}; which will automatically suspend the tracee as
\textbf{ptrace} gives us complete control over the tracee.
So, before the signal is actually received by the tracee, we have control
over the tracee and we can resume it later while suppressing the signal that
should never be delivered as we only send the signal to suspend the tracee and
gain control.

Once the tracee is suspended, we can jump to some previously crafted assembly
similar to our safe fork code (Section \ref{safe-fork}), and execute a system
call by simply invoking the system call instruction and modifying the
arguments and system call instruction in Tracy. Once the system call
has completed we can store the return code and jump back to the original code,
and the finally resume the tracee.

The process just described can also be used to inject system calls
in the signal hook exposed by Tracy.

While it would not be too hard to implement support for this particular
feature, we have postponed the feature due to time constraints. We also
did not find the feature particularly useful because Tracy is event based,
which means that in most cases the tracee is already stopped.

% XXX: Dit kan onhandig zijn in combinatie met threaded tracing. Wat als
% we een signal sturen en een andere thread die afvangt?


%\textit{Netjes uitschrijven.}
%\begin{itemize}
%    \item Securing from mixed ABI using dynamic X\^W protection. (Including
%        picture)
%    \item Support more architectures and OSes.
%    \item Sharing memory between tracer and children using mapped files.
%    \item Instant injection of system calls using signals to stop the process.
%    \item Multithreaded tracing.
%    %\item Python bindings
%\end{itemize}

\section{Calling functions in the tracee}

\section{Tracing 32 bit applications with a 64 bit tracer}

\section{Improving ptrace performance}
\label{ptrace-perf}

As previously discussed in Section \ref{ptrace-perf-problems}, the overhead
of \textbf{ptrace} is very noticeable when a tracee performs a lot of
small and inexpensive system calls - they suddenly become very expensive.

The problem boils down to the fact that \textbf{ptrace} currently
has no mechanism which allows the tracer to only get notified on specific
system calls, so that other system calls can execute without requiring
intervention from the tracer.

It then follows that this performance
issue can be solved by adding such an API to the Linux kernel. \\
% TODO: We kunnen SECCOMP referencen.

We wrote a simple Linux kernel patch (against Linux 3.4) to demonstrate
our claim and we also added experimental support for this feature to Tracy.

However, the patch is incomplete and lacks features such as support for
multiple ABIs (Section \ref{secure-abi}). \\

The API is very basic and allows one to add (and delete) specific system
calls to a list. This list can then either function as a whitelist or a
blacklist, in other words, the Linux kernel either notifies the tracer of
all the system calls in the list, or notifies the tracer of all system calls
but the system calls in the list.

% TODO: API verder beschrijven?

Because Tracy is aware of the all the system calls that are hooked by the
program using Tracy, Tracy can make informed decisions about system calls
that do not need to be received - as they would never result in a hook in
the event system (Section \ref{event-system}) being executed.

Unless Tracy uses this particular system call internally, Tracy could decide
to advice the kernel to not notify Tracy of in the event that this particular
system call is executed.

% TODO: Patch in appendix?


%Applications
%\begin{itemize}
%    \item Buggy: Bug generation for application testing.
%    \item Jelly: Secure Jail
%    \item Fussy: User space fuse.
%    \item Trippy: Visualising system calls using audio (and possibly video)
%    \item Rippy: Ripping/dumping media streams.
%    \item Dumpy: I/O dumping. (Like tcpdump)
%    \item Loggy/sTracy/dotty: System call logging/tracing/visualisation.
%\end{itemize}


% \chapter{Related work}
% 
% \section{strace}

% strace
% gdb
% jail?
% ltrace? (is niet ptrace based)


% Discussie is re-capture van alles weg gezegd hebben
\chapter{Discussion}

% Performance sectie / chapter ergens?
% Viability

\bibliography{report}

\pagebreak

\appendix
\addappheadtotoc

\chapter{Creating and tracing a process}
\label{appendix:createtrace}

% XXX: Willen we hier failsafes in?
\begin{lstlisting}
    pid = fork();

    /* Child */
    if (pid == 0) {
        r = ptrace(PTRACE_TRACEME, 0, NULL, NULL);
        if (r) {
            fprintf(stderr, "PTRACE_TRACEME failed.\n");
            _exit(1);
        }

        raise(SIGTRAP);

        execve(...);

        /* ... */
    }

    if (pid == -1)
        return NULL;

    waitpid(pid, &status, __WALL);
    signal_id = WSTOPSIG(status);
    if (signal_id != SIGTRAP) {
        return NULL;
    }

    /* Resume child and tell ptrace to stop at the next system call */
    r = ptrace(PTRACE_SYSCALL, pid, NULL, 0);
\end{lstlisting}

\chapter{Safe fork}

% XXX: Hierin/onder wat comments weghalen die niet passen
% in een paper.
\lstinputlisting{../src/tracy/trampy.c}

\end{document}
